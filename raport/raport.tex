\documentclass{article}
\usepackage[utf8]{inputenc}
\usepackage[breaklinks=true]{hyperref}
\usepackage{breakcites}
\usepackage[polish]{babel}
\usepackage{mdframed}
\usepackage{subfiles}
\usepackage{subcaption}
\usepackage{graphicx}
\usepackage{wrapfig}
\usepackage{amsmath}
\usepackage{amsfonts}
\usepackage{polski}
\hypersetup{
    linktoc=all,     %set to all if you want both sections and subsections linked
    linkcolor=blue,  %choose some color if you want links to stand out
}
% Options for packages loaded elsewhere
\PassOptionsToPackage{unicode}{hyperref}
\PassOptionsToPackage{hyphens}{url}
%
\usepackage{lmodern}
\usepackage{amssymb,amsmath}
\usepackage{ifxetex,ifluatex}
\ifnum 0\ifxetex 1\fi\ifluatex 1\fi=0 % if pdftex
  %\usepackage[T1]{fontenc}
  %\usepackage{textcomp} % provide euro and other symbols
\else % if luatex or xetex
  \usepackage{unicode-math}
  \defaultfontfeatures{Scale=MatchLowercase}
  \defaultfontfeatures[\rmfamily]{Ligatures=TeX,Scale=1}
\fi
% Use upquote if available, for straight quotes in verbatim environments
\IfFileExists{upquote.sty}{\usepackage{upquote}}{}
\IfFileExists{microtype.sty}{% use microtype if available
  \usepackage[]{microtype}
  \UseMicrotypeSet[protrusion]{basicmath} % disable protrusion for tt fonts
}{}
\makeatletter
\@ifundefined{KOMAClassName}{% if non-KOMA class
  \IfFileExists{parskip.sty}{%
    \usepackage{parskip}
  }{% else
    \setlength{\parindent}{0pt}
    \setlength{\parskip}{6pt plus 2pt minus 1pt}}
}{% if KOMA class
  \KOMAoptions{parskip=half}}
\makeatother
\usepackage{xcolor}
\IfFileExists{xurl.sty}{\usepackage{xurl}}{} % add URL line breaks if available
\IfFileExists{bookmark.sty}{\usepackage{bookmark}}{\usepackage{hyperref}}
\hypersetup{
  hidelinks,
  pdfcreator={LaTeX via pandoc}}
\urlstyle{same} % disable monospaced font for URLs
\usepackage{color}
\usepackage{fancyvrb}
\newcommand{\VerbBar}{|}
\newcommand{\VERB}{\Verb[commandchars=\\\{\}]}

\DefineVerbatimEnvironment{Highlighting}{Verbatim}{commandchars=\\\{\}}
% Add ',fontsize=\small' for more characters per line
\newenvironment{Shaded}{}{}
\newcommand{\AlertTok}[1]{\textcolor[rgb]{1.00,0.00,0.00}{\textbf{#1}}}
\newcommand{\AnnotationTok}[1]{\textcolor[rgb]{0.38,0.63,0.69}{\textbf{\textit{#1}}}}
\newcommand{\AttributeTok}[1]{\textcolor[rgb]{0.49,0.56,0.16}{#1}}
\newcommand{\BaseNTok}[1]{\textcolor[rgb]{0.25,0.63,0.44}{#1}}
\newcommand{\BuiltInTok}[1]{#1}
\newcommand{\CharTok}[1]{\textcolor[rgb]{0.25,0.44,0.63}{#1}}
\newcommand{\CommentTok}[1]{\textcolor[rgb]{0.38,0.63,0.69}{\textit{#1}}}
\newcommand{\CommentVarTok}[1]{\textcolor[rgb]{0.38,0.63,0.69}{\textbf{\textit{#1}}}}
\newcommand{\ConstantTok}[1]{\textcolor[rgb]{0.53,0.00,0.00}{#1}}
\newcommand{\ControlFlowTok}[1]{\textcolor[rgb]{0.00,0.44,0.13}{\textbf{#1}}}
\newcommand{\DataTypeTok}[1]{\textcolor[rgb]{0.56,0.13,0.00}{#1}}
\newcommand{\DecValTok}[1]{\textcolor[rgb]{0.25,0.63,0.44}{#1}}
\newcommand{\DocumentationTok}[1]{\textcolor[rgb]{0.73,0.13,0.13}{\textit{#1}}}
\newcommand{\ErrorTok}[1]{\textcolor[rgb]{1.00,0.00,0.00}{\textbf{#1}}}
\newcommand{\ExtensionTok}[1]{#1}
\newcommand{\FloatTok}[1]{\textcolor[rgb]{0.25,0.63,0.44}{#1}}
\newcommand{\FunctionTok}[1]{\textcolor[rgb]{0.02,0.16,0.49}{#1}}
\newcommand{\ImportTok}[1]{#1}
\newcommand{\InformationTok}[1]{\textcolor[rgb]{0.38,0.63,0.69}{\textbf{\textit{#1}}}}
\newcommand{\KeywordTok}[1]{\textcolor[rgb]{0.00,0.44,0.13}{\textbf{#1}}}
\newcommand{\NormalTok}[1]{#1}
\newcommand{\OperatorTok}[1]{\textcolor[rgb]{0.40,0.40,0.40}{#1}}
\newcommand{\OtherTok}[1]{\textcolor[rgb]{0.00,0.44,0.13}{#1}}
\newcommand{\PreprocessorTok}[1]{\textcolor[rgb]{0.74,0.48,0.00}{#1}}
\newcommand{\RegionMarkerTok}[1]{#1}
\newcommand{\SpecialCharTok}[1]{\textcolor[rgb]{0.25,0.44,0.63}{#1}}
\newcommand{\SpecialStringTok}[1]{\textcolor[rgb]{0.73,0.40,0.53}{#1}}
\newcommand{\StringTok}[1]{\textcolor[rgb]{0.25,0.44,0.63}{#1}}
\newcommand{\VariableTok}[1]{\textcolor[rgb]{0.10,0.09,0.49}{#1}}
\newcommand{\VerbatimStringTok}[1]{\textcolor[rgb]{0.25,0.44,0.63}{#1}}
\newcommand{\WarningTok}[1]{\textcolor[rgb]{0.38,0.63,0.69}{\textbf{\textit{#1}}}}
\setlength{\emergencystretch}{3em} % prevent overfull lines
\providecommand{\tightlist}{%
  \setlength{\itemsep}{0pt}\setlength{\parskip}{0pt}}


%End of Configuration


\title{Praca nad projektem TPCReco i reorganizacja kodu}
\author{Jakub Żak \and Wojciech Kos \and Agata Bijak \and Michał Pasiowiec}
\date{}
\begin{document}

\maketitle
\vfill
\centerline{\large Opiekun: dr hab. Artur Kalinowski}
\vfill

\pagebreak
\section{Wstęp}
Program \textit{TPCReco} służy do analizy danych z detektora \textit{ELITPC}, który bada reakcje fotonuklearne metodą monochromatycznych promieni gamma. Został on napisany w języku \texttt{C++} i jest aktywnie rozwijany na Wydziale Fizyki UW. Czerpie w dużej mierze z genewskiej biblioteki do analizy danych \textit{ROOT}\cite{cern}.

\section{Cel projektu}
Celem Projektu jest reorganizacja istniejącego kodu do analizy danych z eksperymentu Warsaw Active target TPC (dawniej ELITPC).
Projekt podizelony został na trzy części: 


\begin{itemize}
    \item Stworzenie wzoru testów funkcjonalnych klass aplikacji, 
    \item Reorganizacja konfiguracji aplikacji,
    \item Reorganizacja klasy  EventSource,
\end{itemize}
Projekt wykonano przy pomocy systemu wersjonowania kodu (\textit{GIT}\cite{git}) oraz oprogramowania służącego do konteneryzacji projektu (\textit{Docker}\cite{docker})

\vfill
\pagebreak

\section{Kod projektu}
\label{kod}
Repozytorium projektu znajduje się pod adresem \texttt{https://github.com/Kinexity/TPCReco}. Tam wprowadzano wszelkie zmiany za pomocą programu \textit{GIT}, które na koniec włączono do głównego repozytorium aplikacji pod adresem, \texttt{https://github.com/akalinow/TPCReco}. Uczestnicy pracowali na następujących gałęziach projektu:
\begin{enumerate}
    \item \texttt{develJZ}
    \item \texttt{Zmiiokmoesker}.
\end{enumerate}

\section{Wprowadzone zmiany}
Poniżej wymieniono główne zmiany w kodzie:
\subsection{Test funkcjonalny kalsy \texttt{EventTPC.cpp}}
W pliku testu \texttt{EventTPC\_test.cpp} znajdują się funkcje typu void, które przyjmują następujące argumenty:
\begin{itemize}
    \item
  std::shared\_ptr EventPtr =
  myEventSource-\textgreater loadDataFile(dataFileName)-\textgreater getCurrentEvent()
    \item
  std::map\textless std::string, double\textgreater{} Test\_Reference -
  zdefiniowany w dataEventTPC.h
    \item
  std::map\textless std::string, std::string\textgreater{}
  Test\_Reference\_Titles - zdefiniowany w dataEventTPC.h
    
\end{itemize}
Funkcje testujące to:
\begin{itemize}
    \item get1DProjection\_Titles\_Test 
    \item get2DProjection\_Titles\_Test
    \item get1DProjection\_Test 
    \item get2DProjection\_Test 
    \item GetTotalCharge\_Test 
    \item GetMaxCharge\_Test 
    \item GetMaxChargePos\_Test 
    \item GetSignalRange\_Test 
    \item GetMultiplicity\_Test
\end{itemize}
Powyższe funkcje porównuje wyniki wywołanych metod klasy EventTPC z danymi porównawczymi, które znajdują się w pliku dataEventTPC.h. Dane porównawcze przechowywane są w formacie map, gdzie kluczem są wywoływane metody z argumentami, w formacie std::string.

\subsection{Klasa ConfigManager}
Powstała klasa służąca do wczytywania parametrów z linii poleceń oraz plików konfiguracyjnych, a następnie konfiguracji obiektu boost::property\_tree::ptree i zwracania obiektu skonfigurowanego. Klasa zawiera następujące metody:
\begin{itemize}
    \item boost::property\_tree::ptree getConfig(int, char**) - metoda zwracająca skonfigurowane drzewo
    \item \texttt{boost::program\_options::variables\_map parseCmdLineArgs(int,char)} - metoda do parsowania argumentów z linii poleceń
    \item \texttt{boost::program\_options::options\_description parseAllowedArgs(std::string)} - metoda do parsowania listy dozwolonych argumentów linii poleceń z pliku konfiguracyjnego
\end{itemize}
Sposób formatowania plików konfiguracyjnych zawierających listę dozwolonych argumentów linii poleceń został zawarty w TPCReco/Utilities/config/README.md. Przykładowy plik wygląda następująco:

\begin{Shaded}
\begin{Highlighting}[]
\FunctionTok{\{}   \DataTypeTok{"Options"}\FunctionTok{:}
    \FunctionTok{\{}
        \DataTypeTok{"pressure"}\FunctionTok{:} \FunctionTok{\{}
            \DataTypeTok{"type"}\FunctionTok{:} \StringTok{"float"}\FunctionTok{,}
            \DataTypeTok{"defaultValue"}\FunctionTok{:} \DecValTok{60}\FunctionTok{,}
            \DataTypeTok{"description"}\FunctionTok{:} \StringTok{"float {-} CO2 pressure [mbar]"}\FunctionTok{,}
            \DataTypeTok{"isRequired"}\FunctionTok{:} \KeywordTok{true}
        \FunctionTok{\},}
        \DataTypeTok{"no{-}type"}\FunctionTok{:} \FunctionTok{\{}
            \DataTypeTok{"type"}\FunctionTok{:} \StringTok{"bool"}\FunctionTok{,}
            \DataTypeTok{"defaultValue"}\FunctionTok{:} \KeywordTok{false}\FunctionTok{,}
            \DataTypeTok{"description"}\FunctionTok{:} \StringTok{"Skip comparing event type"}\FunctionTok{,}
            \DataTypeTok{"isRequired"}\FunctionTok{:} \KeywordTok{false}
        \FunctionTok{\},}
        \DataTypeTok{"files"}\FunctionTok{:} \FunctionTok{\{}
            \DataTypeTok{"type"}\FunctionTok{:} \StringTok{"std::vector\textless{}std::string\textgreater{}"}\FunctionTok{,}
            \DataTypeTok{"defaultValue"}\FunctionTok{:} \StringTok{"PLACEHOLDER\_FOR\_VALUE"}\FunctionTok{,}
            \DataTypeTok{"description"}\FunctionTok{:} \StringTok{"strings {-} list of files to browse. Mutually exclusive with \textquotesingle{}directory\textquotesingle{}"}\FunctionTok{,}
            \DataTypeTok{"isRequired"}\FunctionTok{:} \KeywordTok{false}
        \FunctionTok{\}}
    \FunctionTok{\}}
\FunctionTok{\}}
\end{Highlighting}
\end{Shaded}

\subsection{Reorganizacja klasy EventSource}

Wyseparowano system rozpoznawania typu danych z klasy MainFrame. Dodano namespace EventSourceFactory z funkcją EventSourceFactory::makeEventSourceObject która przyjmuje przez referencję obiekt typu
boost::property\_tree::ptree z konfiguracją i zwraca std::shared\_ptr\texttt{<EventSourceBase>} będący wskaźnikiem na obiekt klasy pochodnej do EventSourceBase. Typ tego obiektu zależy od nazw plików z danymi zawartych w konfiguracji. W trakcie tworzenia tego obiektu konfiguracja zostaje rozszerzona o zmienną zapisującą typ obiektu ("eventType") w formacie enum event\_type (wartości odpowiadające nazwom klas EventSource[...]) oraz o flagę ("onlineFlag"), która mówi czy pliki mają być na żywo ładowane z folderu (aplikuje się tylko do plików typu GRAW). Zmieniono obsługę systemu plików z biblioteki ROOT na bibliotekę boost::filesystem. Funkcja jest w stanie rozpoznać i załadować następujące formaty plików:
\begin{itemize}
    \item pliki w formacie .root wygenerowane przez bibliotekę ROOT - klasa EventSourceROOT
    \item pliki z symulacji Monte Carlo - klasa EventSourceMC
    \item pliki w formacie GRAW (zarówno pojedyncze, mnogie jak i z możliwość aktywnego ładowania nowo       dodanych plików) - klasy EventSourceGRAW oraz EventSourceMultiGRAW
\end{itemize}
W przypadku nieprawidłowej konfiguracji zostaje zwrócony pusty wskaźnik.
\section{Podsumowanie}
Projekt wykonano podczas semestru zimowego oraz letniego roku akademickiego 2022/2023 na Wydziale Fizyki UW. W wyniku prac, dodano test kalsy EventTPC, zmieniony został sposób konfiguracji aplikacji oraz poprawione zostało rozpoznawanie obiektów EventSourceBase. Dzięki temu, uczestnicy zrealizowali ogólne założenia projektu.

\pagebreak

\begin{thebibliography}{99}
    \bibitem{cern}
        Rene Brun and Fons Rademakers, ROOT - An Object Oriented Data Analysis Framework,
        Proceedings AIHENP'96 Workshop, Lausanne, Sep. 1996,
        Nucl. Inst. \& Meth. in Phys. Res. A 389 (1997) 81-86.
        See also "ROOT" [software], Release v6.26/02, 13/04/2022,
        (\url{https://root.cern/releases/release-61802/})
    \bibitem{git}
        GitHub is a web-based hosting service for version control using git.
        More information about GitHub can be found at \url{https://en.wikipedia.org/wiki/GitHub}.
    \bibitem{docker}
        Docker is a set of platform as a service products that uses OS-level virtualization to deliver software in packages called containers.
        More informations about Docker can be found at \url{https://en.wikipedia.org/wiki/Docker_(software)}
    \bibitem{JSON}
        JSON syntax - \url{https://en.wikipedia.org/wiki/JSON\#Syntax}
\end{thebibliography}
\end{document}


